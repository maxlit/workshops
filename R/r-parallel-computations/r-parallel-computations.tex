% Created 2016-09-03 sob 14:54
\documentclass[bigger]{beamer}
\usepackage[utf8]{inputenc}
\usepackage[T1]{fontenc}
\usepackage{fixltx2e}
\usepackage{graphicx}
\usepackage{longtable}
\usepackage{float}
\usepackage{wrapfig}
\usepackage{soul}
\usepackage{textcomp}
\usepackage{marvosym}
\usepackage{wasysym}
\usepackage{latexsym}
\usepackage{amssymb}
\usepackage{hyperref}
\tolerance=1000
\usepackage[dvipsnames]{color} 
\mode<beamer>{\usetheme{Hannover}}
\providecommand{\alert}[1]{\textbf{#1}}

\title{R programming language: parallel computations}
\author{Maxim Litvak}
\date{2016-06-10}
\hypersetup{
  pdfkeywords={},
  pdfsubject={},
  pdfcreator={Emacs Org-mode version 7.9.3f}}

\begin{document}

\maketitle

\begin{frame}
\frametitle{Outline}
\setcounter{tocdepth}{3}
\tableofcontents
\end{frame}
\section{Introduction}
\label{sec-1}
\begin{frame}
\frametitle{Introduction}
\label{sec-1-1}
\begin{itemize}

\item A task to produce result Y out of input X may have steps that could be performed in parallel
\label{sec-1-1-1}%

\item Example: you need to drink a 1L of beer
\label{sec-1-1-2}%
\begin{itemize}

\item Ask a friend to help
\label{sec-1-1-2-1}%

\item The taks is, thus, parallelized among two throats
\label{sec-1-1-2-2}%
\end{itemize} % ends low level
\end{itemize} % ends low level
\end{frame}
\begin{frame}
\frametitle{Who does the job}
\label{sec-1-2}
\begin{itemize}

\item you can split the tasks in parallel steps, but who does the job?
\label{sec-1-2-1}%

\item Nowadays, each PC usually has 2-8 CPU cores.
\label{sec-1-2-2}%
\begin{itemize}

\item Each core can process a task!
\label{sec-1-2-2-1}%
\end{itemize} % ends low level

\item You can split the tasks across different computers
\label{sec-1-2-3}%
\begin{itemize}

\item You can rent them at Microsoft, Amazon etc
\label{sec-1-2-3-1}%
\end{itemize} % ends low level
\end{itemize} % ends low level
\end{frame}
\begin{frame}
\frametitle{How it works - MapReduce framework}
\label{sec-1-3}
\begin{itemize}

\item Map step - assign to each work a task to perform
\label{sec-1-3-1}%

\item Reduce step - collect results from different workers and produce the overall result
\label{sec-1-3-2}%
\end{itemize} % ends low level
\end{frame}
\section{Loss distribution computation}
\label{sec-2}
\begin{frame}
\frametitle{Basics}
\label{sec-2-1}
\begin{itemize}

\item Random numbers used for simulations are ``pseudo-random'', i.e. a sequence of number that looks random
\label{sec-2-1-1}%

\item The starting point is called a ``seed''. Same seed - same sequence
\label{sec-2-1-2}%

\item Any simulation system must be reproducible
\label{sec-2-1-3}%
\end{itemize} % ends low level
\end{frame}
\begin{frame}
\frametitle{Simulation}
\label{sec-2-2}
\begin{itemize}

\item assume we need to calculate 10 million of random numbers
\label{sec-2-2-1}%

\item if we can split it among 4 CPU cores than each of them needs to generate only 2.5 millions
\label{sec-2-2-2}%

\item we can assign to each core a different seed, generate the numbers and unite the numbers into one
\label{sec-2-2-3}%
\end{itemize} % ends low level
\end{frame}
\section{Parallel computation in R}
\label{sec-3}
\begin{frame}[fragile]
\frametitle{Implementation - I}
\label{sec-3-1}
\begin{itemize}

\item Loss distribution - non-parallel\\
\label{sec-3-1-1}%
\begin{verbatim}
loss.dist <- function(seed, N)
{
  set.seed(seed)
  return(runif(N))
}

N <- 4
seed <- 1
print(loss.dist(seed, N))
\end{verbatim}

\begin{verbatim}
 [1] 0.2655087 0.3721239 0.5728534 0.9082078
\end{verbatim}

\end{itemize} % ends low level
\end{frame}
\begin{frame}[fragile]
\frametitle{Implementation - II}
\label{sec-3-2}
\begin{itemize}

\item Explore your ressources\\
\label{sec-3-2-1}%
\begin{verbatim}
library(parallel)
print(detectCores())
\end{verbatim}

\begin{verbatim}
 [1] 4
\end{verbatim}


\item Loss distribution - parallel\\
\label{sec-3-2-2}%
\begin{verbatim}
N <- 4
seed <- 1
print(loss.dist(seed, N))
\end{verbatim}


\end{itemize} % ends low level
\end{frame}
\begin{frame}
\frametitle{Using library parallel}
\label{sec-3-3}
\begin{itemize}

\item A lot of implementation details are hidden in R (e.g. compared to C\#)
\label{sec-3-3-1}%

\item General schema
\label{sec-3-3-2}%
\begin{itemize}

\item Create a cluster (\textbf{makeCluster})
\label{sec-3-3-2-1}%

\item Put in the scope of cluster objects (functions, variables etc) which are needed there (\textbf{clusterExport})
\label{sec-3-3-2-2}%

\item Send a task (function together with input) to the cluster (\textbf{parLapply}) - similar to apply functions
\label{sec-3-3-2-3}%
\end{itemize} % ends low level
\end{itemize} % ends low level
\end{frame}
\begin{frame}
\frametitle{Mean calculation}
\label{sec-3-4}
\begin{itemize}

\item Take a trivial task as an example - calculate sum of the sample
\label{sec-3-4-1}%

\item divide N numbers sample on M workers
\label{sec-3-4-2}%

\item calculate the sums for each workers (in parallel)
\label{sec-3-4-3}%

\item collect the results and calculate their sum
\label{sec-3-4-4}%
\end{itemize} % ends low level
\end{frame}
\begin{frame}[fragile]
\frametitle{Sum parallel calculation - hands-on}
\label{sec-3-5}


\begin{verbatim}
nc <- 2 # number of cores
cl <- makeCluster(no_cores)
sq <- 1:8
L <- length(sq)
clusterExport(cl, list("sq", "nc", "L"))
# check first that the split is correct
parLapply(cl, 1:nc
  , function(x) sq[(1+(x-1)*L/nc):(x*L/nc)]
        )
# [[1]] 
# [1] 1 2 3 4
# [[2]]
# [1] 5 6 7 8
\end{verbatim}
\end{frame}
\begin{frame}[fragile]
\frametitle{Sum parallel calculation - hands-on II}
\label{sec-3-6}


\begin{verbatim}
res <- parLapply(cl, 1:nc
, function(x) sum(sq[(1+(x-1)*L/nc):(x*L/nc)])
        )
print(res)
# [[1]]
# [1] 10
# [[2]]
# [1] 26
print(sum(unlist(res))) # collect results
# [1] 36
\end{verbatim}
\end{frame}
\begin{frame}
\frametitle{Example - wrong implementation}
\label{sec-3-7}
\begin{itemize}

\item Farenheit to Celcius
\label{sec-3-7-1}%
\end{itemize} % ends low level
\end{frame}
\begin{frame}[fragile]
\frametitle{Task implement Farenheit to Celcius transform in parallel}
\label{sec-3-8}
\begin{itemize}

\item Take this as input (correct \textbf{parLapply} call is to be implemented)\\
\label{sec-3-8-1}%
\begin{verbatim}
library(parallel)
c <- function(t) t*5/9-32
nc <- 4
temps <- seq(10, 40, 10)
cl <- makeCluster(nc)
clusterExport(cl, list("temps"))
\end{verbatim}
\end{itemize} % ends low level
\end{frame}
\begin{frame}
\frametitle{Example - wrong implementation}
\label{sec-3-9}
\begin{itemize}

\item Example attempt\\
\label{sec-3-9-1}%
\emph{parLapply(cl, 1:nc, function(x) c(temps[x]))}

\item However, temperatures are still in Farenheit, what is wrong here?
\label{sec-3-9-2}%

\item Try to correct it
\label{sec-3-9-3}%
\end{itemize} % ends low level
\end{frame}
\begin{frame}[fragile]
\frametitle{Example - correction}
\label{sec-3-10}


\begin{verbatim}
library(parallel)
C <- function(t) t*5/9-32
nc <- 4
temps <- seq(10, 40, 10)
cl <- makeCluster(nc)
clusterExport(cl, list("temps", "C"))
parLapply(cl, 1:nc, function(x) C(temps[x]))
\end{verbatim}


\begin{verbatim}
[[1]]
[1] -26.44444

[[2]]
[1] -20.88889

[[3]]
[1] -15.33333

[[4]]
[1] -9.777778
\end{verbatim}
\end{frame}
\begin{frame}
\frametitle{Example - wrong implementation II}
\label{sec-3-11}
\begin{itemize}

\item standard object \textbf{c} was overwritten and wasn't put in the scope of cluster
\label{sec-3-11-1}%

\item Even worse it didn't throw an error since the cluster used the default object
\label{sec-3-11-2}%
\end{itemize} % ends low level
\end{frame}
\begin{frame}
\frametitle{Pay attention}
\label{sec-3-12}
\begin{itemize}

\item Be careful with the scope of the cluster
\label{sec-3-12-1}%

\item distribute carefully among work loaders
\label{sec-3-12-2}%
\end{itemize} % ends low level
\end{frame}
\begin{frame}[fragile]
\frametitle{Parallel loss distribution calculation}
\label{sec-3-13}
\begin{itemize}

\item consider a simple loss generation\\
\label{sec-3-13-1}%
\begin{verbatim}
loss.dist <- function(seed, N)
{
  set.seed(seed)
  return(runif(N))
}
print(loss.dist(1,4))
# [1] 0.2655087 0.3721239 0.5728534 0.9082078
\end{verbatim}

\begin{verbatim}
 [1] 0.2655087 0.3721239 0.5728534 0.9082078
\end{verbatim}


\item How to parallelize it?
\label{sec-3-13-2}%
\end{itemize} % ends low level
\end{frame}
\begin{frame}[fragile]
\frametitle{Parallel loss calculation}
\label{sec-3-14}
\begin{itemize}

\item Possible solution\\
\label{sec-3-14-1}%
\begin{verbatim}
parallel.loss.dist <- function(seed, N)
{
  no_cores <- 2
  cl <- makeCluster(no_cores)
  clusterExport(cl, list("loss.dist"))
  temp.res <- parLapply(
        cl
        , seed:(seed + 1)
        , function(x) loss.dist(x, N = 3)
        )
  stopCluster(cl)
  return(unlist(temp.res))
}
print(parallel.loss.dist(1,4))
# [1] 0.2655087 0.3721239 0.1848823 0.7023740
\end{verbatim}


\end{itemize} % ends low level
\end{frame}
\begin{frame}
\frametitle{Questions}
\label{sec-3-15}
\begin{itemize}

\item Why the first 2 numbers coincide with non-parallel version and the rest not?
\label{sec-3-15-1}%

\item Where is the ``Map'' step and where is the ``Reduce'' step hidden in the code?
\label{sec-3-15-2}%
\end{itemize} % ends low level
\end{frame}
\begin{frame}[fragile]
\frametitle{Time measurement}
\label{sec-3-16}
\begin{itemize}

\item Let's measure time with the following function (not optimal)\\
\label{sec-3-16-1}%
\begin{verbatim}
measure.time <- function(command)
{
  start.time <- Sys.time()
  eval(parse(text = command))
  end.time <- Sys.time()
  d <- difftime(end.time, start.time
        , units = "secs"))
  return(d)
}
# Example:
# cmd <- "sum(parallel.loss.dist(1, 1e+08))"
# measure.time(cmd)
\end{verbatim}


\end{itemize} % ends low level
\end{frame}
\begin{frame}
\frametitle{Time measurement II}
\label{sec-3-17}
\begin{itemize}

\item play with N to see when it pays off to use parallel or non-parallel version
\label{sec-3-17-1}%

\item adjust additionally the number of cores - does it get faster?
\label{sec-3-17-2}%
\end{itemize} % ends low level
\end{frame}

\end{document}
