% Created 2017-07-15 Sat 12:18
\documentclass[bigger]{beamer}
\usepackage[utf8]{inputenc}
\usepackage[T1]{fontenc}
\usepackage{fixltx2e}
\usepackage{graphicx}
\usepackage{longtable}
\usepackage{float}
\usepackage{wrapfig}
\usepackage{rotating}
\usepackage[normalem]{ulem}
\usepackage{amsmath}
\usepackage{textcomp}
\usepackage{marvosym}
\usepackage{wasysym}
\usepackage{amssymb}
\usepackage{hyperref}
\tolerance=1000
\usetheme{default}
\author{Maxim Litvak}
\date{2016-06-10}
\title{R programming language: parallel computations}
\hypersetup{
  pdfkeywords={},
  pdfsubject={},
  pdfcreator={Emacs 25.2.1 (Org mode 8.2.10)}}
\begin{document}

\maketitle
\begin{frame}{Outline}
\tableofcontents
\end{frame}

\section{Introduction}
\label{sec-1}
\subsection{Introduction}
\label{sec-1-1}
\begin{frame}[label=sec-1-1-1]{A task to produce result Y out of input X may have steps that could be performed in parallel}
\end{frame}
\begin{frame}[label=sec-1-1-2]{Example: you need to drink a 1L of beer}
\begin{block}{Ask a friend to help}
\end{block}
\begin{block}{The taks is, thus, parallelized among two throats}
\end{block}
\end{frame}
\subsection{Who does the job}
\label{sec-1-2}
\begin{frame}[label=sec-1-2-1]{you can split the tasks in parallel steps, but who does the job?}
\end{frame}
\begin{frame}[label=sec-1-2-2]{Nowadays, each PC usually has 2-8 CPU cores.}
\begin{block}{Each core can process a task!}
\end{block}
\end{frame}
\begin{frame}[label=sec-1-2-3]{You can split the tasks across different computers}
\begin{block}{You can rent them at Microsoft, Amazon etc}
\end{block}
\end{frame}
\subsection{How it works - MapReduce framework}
\label{sec-1-3}
\begin{frame}[label=sec-1-3-1]{Map step - assign to each work a task to perform}
\end{frame}
\begin{frame}[label=sec-1-3-2]{Reduce step - collect results from different workers and produce the overall result}
\end{frame}
\section{Loss distribution computation}
\label{sec-2}
\subsection{Basics}
\label{sec-2-1}
\begin{frame}[label=sec-2-1-1]{Random numbers used for simulations are "pseudo-random", i.e. a sequence of number that looks random}
\end{frame}
\begin{frame}[label=sec-2-1-2]{The starting point is called a "seed". Same seed - same sequence}
\end{frame}
\begin{frame}[label=sec-2-1-3]{Any simulation system must be reproducible}
\end{frame}
\subsection{Simulation}
\label{sec-2-2}
\begin{frame}[label=sec-2-2-1]{assume we need to calculate 10 million of random numbers}
\end{frame}
\begin{frame}[label=sec-2-2-2]{if we can split it among 4 CPU cores than each of them needs to generate only 2.5 millions}
\end{frame}
\begin{frame}[label=sec-2-2-3]{we can assign to each core a different seed, generate the numbers and unite the numbers into one}
\end{frame}
\section{Parallel computation in R}
\label{sec-3}
\subsection{Implementation - I}
\label{sec-3-1}
\begin{frame}[fragile,label=sec-3-1-1]{Loss distribution - non-parallel}
 \begin{verbatim}
loss.dist <- function(seed, N)
{
  set.seed(seed)
  return(runif(N))
}

N <- 4
seed <- 1
print(loss.dist(seed, N))
\end{verbatim}

\begin{verbatim}
[1] 0.2655087 0.3721239 0.5728534 0.9082078
\end{verbatim}
\end{frame}

\subsection{Implementation - II}
\label{sec-3-2}
\begin{frame}[fragile,label=sec-3-2-1]{Explore your ressources}
 \begin{verbatim}
library(parallel)
print(detectCores())
\end{verbatim}

\begin{verbatim}
[1] 8
\end{verbatim}
\end{frame}

\begin{frame}[fragile,label=sec-3-2-2]{Loss distribution - parallel}
 \begin{verbatim}
N <- 4
seed <- 1
print(loss.dist(seed, N))
\end{verbatim}
\end{frame}

\subsection{Using library parallel}
\label{sec-3-3}
\begin{frame}[label=sec-3-3-1]{A lot of implementation details are hidden in R (e.g. compared to C\#)}
\end{frame}
\begin{frame}[label=sec-3-3-2]{General schema}
\begin{block}{Create a cluster (\alert{makeCluster})}
\end{block}
\begin{block}{Put in the scope of cluster objects (functions, variables etc) which are needed there (\alert{clusterExport})}
\end{block}
\begin{block}{Send a task (function together with input) to the cluster (\alert{parLapply}) - similar to apply functions}
\end{block}
\end{frame}
\subsection{Sum calculation}
\label{sec-3-4}
\begin{frame}[label=sec-3-4-1]{Take a trivial task as an example - calculate sum of the sample}
\end{frame}
\begin{frame}[label=sec-3-4-2]{divide N numbers sample on M workers}
\end{frame}
\begin{frame}[label=sec-3-4-3]{calculate the sums for each workers (in parallel)}
\end{frame}
\begin{frame}[label=sec-3-4-4]{collect the results and calculate their sum}
\end{frame}
\subsection{Sum parallel calculation - hands-on}
\label{sec-3-5}
\begin{verbatim}
nc <- 2 # number of cores
cl <- makeCluster(no_cores)
sq <- 1:8
L <- length(sq)
clusterExport(cl, list("sq", "nc", "L"))
# check first that the split is correct
parLapply(cl, 1:nc
  , function(x) sq[(1+(x-1)*L/nc):(x*L/nc)]
	)
# [[1]] 
# [1] 1 2 3 4
# [[2]]
# [1] 5 6 7 8
\end{verbatim}

\subsection{Sum parallel calculation - hands-on II}
\label{sec-3-6}
\begin{verbatim}
res <- parLapply(cl, 1:nc
, function(x) sum(sq[(1+(x-1)*L/nc):(x*L/nc)])
	)
print(res)
# [[1]]
# [1] 10
# [[2]]
# [1] 26
print(sum(unlist(res))) # collect results
# [1] 36
\end{verbatim}

\subsection{Example - wrong implementation}
\label{sec-3-7}
\begin{frame}[label=sec-3-7-1]{Farenheit to Celcius}
\end{frame}
\subsection{Task implement Farenheit to Celcius transform in parallel}
\label{sec-3-8}
\begin{frame}[fragile,label=sec-3-8-1]{Take this as input (correct \alert{parLapply} call is to be implemented)}
 \begin{verbatim}
library(parallel)
c <- function(t) t*5/9-32
nc <- 4
temps <- seq(10, 40, 10)
cl <- makeCluster(nc)
clusterExport(cl, list("temps"))
\end{verbatim}
\end{frame}
\subsection{Example - wrong implementation}
\label{sec-3-9}
\begin{frame}[label=sec-3-9-1]{Example attempt}
\emph{parLapply(cl, 1:nc, function(x) c(temps[x]))}
\end{frame}
\begin{frame}[label=sec-3-9-2]{However, temperatures are still in Farenheit, what is wrong here?}
\end{frame}
\begin{frame}[label=sec-3-9-3]{Try to correct it}
\end{frame}
\subsection{Example - correction}
\label{sec-3-10}
\begin{verbatim}
library(parallel)
C <- function(t) t*5/9-32
nc <- 4
temps <- seq(10, 40, 10)
cl <- makeCluster(nc)
clusterExport(cl, list("temps", "C"))
parLapply(cl, 1:nc, function(x) C(temps[x]))
\end{verbatim}

\begin{verbatim}
[[1]]
[1] -26.44444

[[2]]
[1] -20.88889

[[3]]
[1] -15.33333

[[4]]
[1] -9.777778
\end{verbatim}

\subsection{Example - wrong implementation II}
\label{sec-3-11}
\begin{frame}[label=sec-3-11-1]{standard object \alert{c} was overwritten and wasn't put in the scope of cluster}
\end{frame}
\begin{frame}[label=sec-3-11-2]{Even worse it didn't throw an error since the cluster used the default object}
\end{frame}
\begin{frame}[label=sec-3-11-3]{Recommended to review the topic on scope in}
\url{https://github.com/maxlit/workshops/blob/master/R/r-advanced-overview/r-workshop-general-overview.pdf}
\end{frame}
\subsection{Pay attention}
\label{sec-3-12}
\begin{frame}[label=sec-3-12-1]{Be careful with the scope of the cluster}
\end{frame}
\begin{frame}[label=sec-3-12-2]{distribute carefully among work loaders}
\end{frame}
\subsection{Parallel loss distribution calculation}
\label{sec-3-13}
\begin{frame}[fragile,label=sec-3-13-1]{consider a simple loss generation}
 \begin{verbatim}
loss.dist <- function(seed, N)
{
  set.seed(seed)
  return(runif(N))
}
print(loss.dist(1,4))
# [1] 0.2655087 0.3721239 0.5728534 0.9082078
\end{verbatim}

\begin{verbatim}
[1] 0.2655087 0.3721239 0.5728534 0.9082078
\end{verbatim}
\end{frame}

\begin{frame}[label=sec-3-13-2]{How to parallelize it?}
\end{frame}
\subsection{Parallel loss calculation}
\label{sec-3-14}
\begin{frame}[fragile,label=sec-3-14-1]{Possible solution}
 \begin{verbatim}
parallel.loss.dist <- function(seed, N)
{
  no_cores <- 2
  cl <- makeCluster(no_cores)
  clusterExport(cl, list("loss.dist"))
  temp.res <- parLapply(
	cl
	, seed:(seed + 1)
	, function(x) loss.dist(x, N)
	)
  stopCluster(cl)
  return(unlist(temp.res))
}
print(parallel.loss.dist(1,4))
# [1] 0.2655087 0.3721239 0.1848823 0.7023740
\end{verbatim}
\end{frame}

\subsection{Questions}
\label{sec-3-15}
\begin{frame}[label=sec-3-15-1]{Why the first 2 numbers coincide with non-parallel version and the rest not?}
\end{frame}
\begin{frame}[label=sec-3-15-2]{Where is the "Map" step and where is the "Reduce" step hidden in the code?}
\end{frame}
\subsection{Time measurement}
\label{sec-3-16}
\begin{frame}[fragile,label=sec-3-16-1]{Let's measure time with the following function (not optimal)}
 \begin{verbatim}
measure.time <- function(command)
{
  start.time <- Sys.time()
  eval(parse(text = command))
  end.time <- Sys.time()
  d <- difftime(end.time, start.time
	, units = "secs"))
  return(d)
}
# Example:
# cmd <- "sum(parallel.loss.dist(1, 1e+08))"
# measure.time(cmd)
\end{verbatim}
\end{frame}

\subsection{Time measurement II}
\label{sec-3-17}
\begin{frame}[label=sec-3-17-1]{play with N to see when it pays off to use parallel or non-parallel version}
\end{frame}
\begin{frame}[label=sec-3-17-2]{adjust additionally the number of cores - does it get faster?}
\end{frame}
\subsection{End}
\label{sec-3-18}
\begin{frame}[label=sec-3-18-1]{Thank you for your attention}
\end{frame}
\begin{frame}[label=sec-3-18-2]{You can find the presentation and the code that was used here at}
\texttt{github.com/maxlit/workshops/tree/master/R/r-parallel-computations}
\end{frame}
% Emacs 25.2.1 (Org mode 8.2.10)
\end{document}
