% Created 2017-05-04 Thu 15:21
\documentclass[presentation]{beamer}
\usepackage[utf8]{inputenc}
\usepackage[T1]{fontenc}
\usepackage{fixltx2e}
\usepackage{graphicx}
\usepackage{longtable}
\usepackage{float}
\usepackage{wrapfig}
\usepackage{rotating}
\usepackage[normalem]{ulem}
\usepackage{amsmath}
\usepackage{textcomp}
\usepackage{marvosym}
\usepackage{wasysym}
\usepackage{amssymb}
\usepackage{hyperref}
\tolerance=1000
\usetheme{Madrid}
\author{Maxim Litvak}
\date{2017-03-06}
\title{Numerics in Applied Mathematical Finance (with R)}
\hypersetup{
  pdfkeywords={},
  pdfsubject={},
  pdfcreator={Emacs 25.1.1 (Org mode 8.2.10)}}
\begin{document}

\maketitle
\begin{frame}{Outline}
\tableofcontents
\end{frame}

\section{Outline}
\label{sec-1}
\begin{frame}[label=sec-1-1]{Outline}
\begin{block}{Numerical derivatives}
\end{block}
\begin{block}{Matrix decomposition}
\end{block}
\begin{block}{Pseudorandom number generators}
\end{block}
\begin{block}{Generation of correlated random numbers}
\end{block}
\end{frame}
\section{Basic numerics}
\label{sec-2}
\begin{frame}[label=sec-2-1]{Numerical derivative - I}
\begin{block}{2 point symmetric derivative}
$f'_{num}(x) = \frac{f(x + h) - f(x - h)}{2h}$
\end{block}
\begin{block}{Both advantage and disadvantage of symmetry:}
More derivatives exist (e.g. modulus function), but some of them you may not want to exist
\end{block}
\begin{block}{Question: how to set h?}
Set h depending on application cases. 
\end{block}
\end{frame}
\begin{frame}[fragile,label=sec-2-2]{Numerical derivative - II}
 \begin{block}{Code example}
\begin{verbatim}
num.deriv <- function(f, x, h = 1e-05)
{
  return((f(x + h) - f(x - h))/(2*h))
}

print(num.deriv(sqrt, 4, .1))
print(num.deriv(sqrt, 4, .01))
\end{verbatim}

\begin{verbatim}
[1] 0.2500195
[1] 0.2500002
\end{verbatim}
\end{block}
\end{frame}

\begin{frame}[label=sec-2-3]{Numerical derivative - III}
\begin{block}{5 point derivative}
If 2-point derivative behaves badly, try more precision
$f'_{num} = \frac{-f(x+2h) + 8f(x+h) - 8f(x-h) + f(x-2h)}{12h}$
\end{block}
\end{frame}
\begin{frame}[label=sec-2-4]{Semidefinite matrix decomposition and eigenvalues}
\begin{block}{Example}
Assume that the default rates in different industries are correlated. The corresponding correlation matrix is positive semidefinite.
\end{block}
\begin{block}{In order to draw the correlated random variables we need to decompose the correlation matrix}
\end{block}
\begin{block}{Possibilities}
\begin{block}{Cholesky decomposition}
$\Sigma = LL^T$ where L is low-triangular
\end{block}
\begin{block}{Eigendecomposition}
$\Sigma = (Q\Lambda^{\frac{1}{2}})(Q\Lambda^{\frac{1}{2}})^T$ where $\Lambda$ is matrix with eigenvalues on diagonal and 0 elsewhere, Q is matrix of eigenvectors
\end{block}
\end{block}
\end{frame}
\section{Numerics in Stochastics}
\label{sec-3}
\begin{frame}[label=sec-3-1]{Pseudo-Random Numbers Generators (PRNG) - I}
\begin{block}{Importance of reproducibility of results}
\begin{block}{Sounds like a paradox: random numbers must be reproducible}
Helps to spot the effects of other factors, e.g. you need to calculate impact of the new pricing algorithm and eliminate effect of randomness.
\end{block}
\end{block}
\begin{block}{Setting seeds}
\begin{itemize}
\item "Whatever one sows, that will he also reap"
\item Input: one number (called "seed"), output: the sequence of numbers that repeats only after very big period
\item Period of currently popular Mersenne Twister is $2^{19937} - 1$
\item RAND() function in Excel2003 has period of $10^{13}$
\item However, there are still legacy systems in use with small period (e.g. 40 mln)
\end{itemize}
\end{block}
\end{frame}
\begin{frame}[fragile,label=sec-3-2]{Pseudo-Random Numbers Generators (PRNG) - II}
 \begin{block}{Code example}
\begin{verbatim}
loss.dist <- function(seed, N)
{
  set.seed(seed)
  return(runif(N))
}

print(loss.dist(seed=1, N=4))
print(loss.dist(seed=1, N=4))  # same seed - same sequence
print(loss.dist(seed=10, N=4)) # different seed - different sequency
\end{verbatim}

\begin{verbatim}
[1] 0.2655087 0.3721239 0.5728534 0.9082078
[1] 0.2655087 0.3721239 0.5728534 0.9082078
[1] 0.5074782 0.3067685 0.4269077 0.6931021
\end{verbatim}
\end{block}
\end{frame}

\begin{frame}[fragile,label=sec-3-3]{Correlated numbers generation}
 \begin{block}{Practical scenario:}
There's a correlation matrix given. However, an expert sets some of the negative correlations to 0 (reality check).
We need to know if the adjusted matrix is still positive semidefinite.
\end{block}
\begin{block}{Approach}
The smallest eigenvalue must be positive.
\end{block}
\begin{block}{Code example}
\begin{verbatim}
R <- matrix(c(1,.5,.5,1), nrow = 2)
print(min(eigen(R)$value))
\end{verbatim}

\begin{verbatim}
          [,1]      [,2]
[1,] 1.0000000 0.5076308
[2,] 0.5076308 1.0000000
\end{verbatim}
\end{block}
\end{frame}

\begin{frame}[fragile,label=sec-3-4]{Correlated numbers generation - II}
 \begin{block}{Refresher: fact from the probability theory}
Let $\xi \in \Phi_{0_n, I_n}$ and $\Sigma = AA^T$
Then $A\xi \in \Phi_{0_n, \Sigma}$
\end{block}
\begin{block}{Code example}
\begin{verbatim}
R <- matrix(c(1,.5,.5,1), nrow = 2)
EG <- eigen(R)
mx <- EG$vectors %*% diag(sqrt(EG$values))
V <- matrix(rnorm(1000), nrow = 2)
print(cor(t(mx%*%V)))
\end{verbatim}
\end{block}
\end{frame}
\begin{frame}[label=sec-3-5]{Computation of quantile functions - I}
\begin{block}{Given}
\begin{itemize}
\item Given: F() - cdf, probability \emph{y}
\item Find: quantile \emph{x}, s.t. y = F(x)
\end{itemize}
\end{block}
\begin{block}{No closed form solution examples}
\begin{itemize}
\item Normal distribution (not even cdf is given in elementary functions!)
\item Gamma distribution
\end{itemize}
\end{block}
\begin{block}{Quantile function given}
If the quantile function is given, it's better to used its Taylor expansion
\end{block}
\begin{block}{Example}
Normal cdf is implemented in practice as a piecewise Taylor polynom, i.e. with coefficients varying on different intervals.
\end{block}
\end{frame}
\begin{frame}[label=sec-3-6]{Computation of quantile functions - II}
\begin{block}{Problem}
Many algorithms require an interval to be defined, however, the quantile function are often defined on unconstrained intervals.
\end{block}
\begin{block}{Example}
\begin{itemize}
\item Find a quantile for Gamma distribution
\item Problem: the right end of domain is unconstrained, additionally, the root finding algorithm might not converge in the tail
\item Solution: use Chebyshev's inequality to constrain the domain
\end{itemize}
\end{block}
\end{frame}
\begin{frame}[label=sec-3-7]{Computation of quantile functions - III}
\begin{block}{Application}
\begin{block}{Inequality}
$P(|X - \mu| \geq 10\sigma) = 0.01$
\end{block}
\begin{block}{In numbers}
\begin{itemize}
\item Default rate 2\%, $\theta = 1$
\item $P(|X - 0.02| \geq 10\times 0.02) = 0.01$
\item Thus, right bound 0.22. If x is bigger than 0.22, then set it hard to 0.99 (if precision in the tail is not important)
\end{itemize}
\end{block}
\end{block}
\begin{block}{Inequality application}
$P(|X - \mu| \geq 10\sigma) = 0.01$
\end{block}
\end{frame}
\begin{frame}[fragile,label=sec-3-8]{Computation of quantile functions - IV}
 \begin{block}{Problem}
Find quantile of Gamma distribution using uniroot procedure
\end{block}
\begin{block}{Solution}
\begin{verbatim}
pg <- function(x) pgamma(x, 0.02, 1) - 0.95
uniroot(pg, c(0,0.22))$root
# check: qgamma(0.95, 0.02, 1)
\end{verbatim}

\begin{verbatim}
[1] 0.04592691
\end{verbatim}
\end{block}
\end{frame}

\section{Application case}
\label{sec-4}
\begin{frame}[label=sec-4-1]{Case - Constrained distribution}
\begin{itemize}
\item Given: normal distribution is used to simulate the collateral value
\item Data: the mean and standard deviation are used from historical observations
\item Proposal: set the negative outcomes to 0
\end{itemize}
\end{frame}
\begin{frame}[label=sec-4-2]{Case - Constrained distribution}
\begin{block}{Problem}
\begin{itemize}
\item the mean and standard deviation would shift after cutting
\item i.e. we need to figure out new parameters, s.t. they would give the historical mean and deviation after truncation
\end{itemize}
\end{block}
\begin{block}{Solution}
\begin{itemize}
\item calculate new mean and standard deviations
\item luckily, the mean and deviation have a close-form solution
\item $\hat{\mu} = \mu + \eta$
\item $\hat{\sigma}^2 = \sigma^2 - \mu\eta + \eta^2$
\item where $\eta = \sigma \frac{\phi(-\frac{\mu}{\sigma})}{1 - \Phi(-\frac{\mu}{\sigma})}$
\item However, no straight-forward way to invert (and we have a system of two equations - one for mean and one for deviation)
\end{itemize}
\end{block}
\end{frame}
\begin{frame}[label=sec-4-3]{Case - Constrained distribution}
\begin{block}{Numerical inversion of a sysdtem of equations}
\begin{itemize}
\item One of possible options is to minimize the sum of error squares
\item $(\hat{\mu}(\mu,\sigma) - \mu_0)^2 + (\hat{\sigma}(\mu,\sigma) - \sigma_0)^2 \rightarrow \min_{\mu,\sigma}$
\item require a multi-dimensional optimization (e.g. gradient descent)
\end{itemize}
\end{block}
\end{frame}
\begin{frame}[label=sec-4-4]{Case - Constrained distribution}
\begin{block}{Some conclusions: a simple (and meaningful) requirement led to:}
\begin{itemize}
\item Mathematical calculations (still feasible)
\item Multivariate optimization (with some numerical tinkering)
\end{itemize}
\end{block}
\end{frame}
% Emacs 25.1.1 (Org mode 8.2.10)
\end{document}
